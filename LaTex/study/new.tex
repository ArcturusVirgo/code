\documentclass[UTF-8]{ctexart}
\usepackage{geometry}  % 调整页边距的包

\geometry{left=2cm,right=2cm}  % 调整页边距
\everymath{\displaystyle}  % 所有公式都使用行间公式大小
\begin{document}

\section{理论力学第一章总结}
\subsection{运动学方程}
\begin{enumerate}
    \item 矢量形式:
    \item 直角坐标:$
              \left \{ \begin{array}{l}
                  x=x(t) \\
                  y=y(t) \\
                  z=z(t) \\
              \end{array} \right.
          $
    \item 极坐标:$
              \left \{ \begin{array}{l}
                  r=r(t)              \\
                  \theta = \theta (t) \\
              \end{array} \right.
          $
    \item 自然坐标(弧坐标):$
              s = s(t)
          $
    \item 柱坐标:$
              \left \{ \begin{array}{l}
                  r=r(t)              \\
                  \theta = \theta (t) \\
                  z=z(t)              \\
              \end{array} \right.
          $
    \item 球坐标:$
              \left \{ \begin{array}{l}
                  r=r(t)                \\
                  \theta = \theta (t)   \\
                  \varphi = \varphi (t) \\
              \end{array} \right.
          $
\end{enumerate}

\begin{table}[h]
    \begin{tabular}{c|c|c|c}
        \hline
         & 运动学方程 & 速度 & 加速度 \\
        \hline
        矢量形式
         & $
            \overrightarrow{r} = \overrightarrow{r}(t)
        $
         & $
            v=\dot{\overrightarrow{r} }
        $
         & $
            a=\dot{\overrightarrow{v} } =\ddot{\overrightarrow{r} }
        $                             \\
        \hline
        直角坐标
         & $
            \left \{ \begin{array}{lll}
                         x=x(t) \\
                         y=y(t) \\
                         z=z(t) \\
                     \end{array} \right.
        $
         & $
            \left \{ \begin{array}{lll}
                         v_x= \dot{x} \\
                         v_y= \dot{y} \\
                         v_z= \dot{z} \\
                     \end{array} \right.
        $
         & $
            \left \{ \begin{array}{lll}
                         a_x= \ddot{x} \\
                         a_y= \ddot{y} \\
                         a_z= \ddot{z} \\
                     \end{array} \right.
        $                             \\
        \hline
        极坐标
         & $
            \left\{ \begin{array}{ll}
                        r=r(t)              \\
                        \theta = \theta (t) \\
                    \end{array} \right.
        $
         & $
            \left\{ \begin{array}{ll}
                        v_r=\dot{r}                 \\
                        v_{\theta} = r \dot{\theta} \\
                    \end{array} \right.
        $
         & $
            \left\{ \begin{array}{ll}
                        a_r=\ddot{r}-r {\dot{\theta}}^2                   \\
                        a_{\theta} = r \ddot{\theta}+2\dot{r}\dot{\theta} \\
                    \end{array} \right.
        $                             \\
        \hline
        自然坐标
         & $
            s=s(t)
        $
         & $
            v=v_t=\dot{s}
        $
         & $
            \left \{ \begin{array}{ll}
                         a_t=\ddot{s}         \\
                         a_n={v^2 \over \rho} \\
                     \end{array} \right.
        $                             \\
        \hline
        球坐标
         & $
            \left \{ \begin{array}{lll}
                         r=r(t)                \\
                         \theta = \theta (t)   \\
                         \varphi = \varphi (t) \\
                     \end{array} \right.
        $
         & $
            \left \{ \begin{array}{lll}
                         v_r=\dot{r}                               \\
                         v_{\theta}=r\dot{\theta}                  \\
                         v_{\varphi}=r \sin {\theta} \dot{\varphi} \\
                     \end{array} \right.
        $
         & $
            \left \{ \begin{array}{lll}
                     \end{array} \right.
        $                             \\
        \hline
    \end{tabular}
\end{table}

\section{量子力学第二章总结}
\subsection{质点组}
\begin{itemize}

    \item 内力的性质:
          \begin{enumerate}
              \item 质点组内力之和为0
              \item 内力对某点$O$的力矩之和为0
              \item 内力做功之和一般不为0
          \end{enumerate}
    \item 质心:
          \begin{enumerate}
              \item 质心坐标:$ \vec{r}_c=\frac{ \sum\limits_{i=1}^nm_i\vec{r}_i}{m} $
              \item 质心速度:$\vec{v}_c=\frac{d\vec{r}_c}{dt}=\frac{ \sum\limits_{i=1}^n m_i\dot{\vec{r}_i}}{m}=\frac{ \sum\limits_{i=1}^n m_i\vec{v}_i}{m}$
              \item 质心加速度:$\vec{a}_c=\frac{d\vec{c}_c}{dt}=\frac{ \sum\limits_{i=1}^n m_i\dot{\vec{v}_i}}{m}=\frac{ \sum\limits_{i=1}^n m_i\vec{a}_i}{m}$
          \end{enumerate}
\end{itemize}
\subsection{动量定理与动量守恒定律}
\subsubsection{动量定理}
\begin{itemize}
    \item 动量
          \begin{itemize}
              \item 惯性系$S$:$\vec{p}=\frac{ \sum\limits_{i=1}^n m_i\vec{v}_i}{m}$
              \item 非惯性系$S'$:${\vec{p}}\ ' =\frac{ \sum\limits_{i=1}^n m_i{\vec{v}_i}\,'}{m}$
              \item 两者之间的关系:$\vec{p}=m\vec{v}_{O'}+\sum_{i=1}^n m_i {\vec{v}_i}\,'$
          \end{itemize}
    \item 质心坐标系下的动量
          \begin{itemize}
              \item 非惯性系$S'$:$\vec{p}\ '=0$
              \item 惯性系$S$:$\vec{p}=m\vec{v}_c$
          \end{itemize}
    \item 惯性系$S$下的动量定理:$ \frac{d\vec{p}}{dt}=\vec{F}^{(e)} $
    \item 非惯性系$S'$下的动量定理:$\frac{d^*\vec{p}}{dt}=\vec{F}^{(e)}-m\vec{a}_{O'}$
\end{itemize}
\subsubsection{质心运动定理}
质心运动定理:$\vec{F}^{(e)}=m \vec{a}_c$
\subsubsection{质点组动量守恒定律}
质点组动量守恒定律:$\frac{d\vec{p}}{dt}=0$
\subsection{动量矩定理与动量矩守恒定理}
\subsubsection{对惯性系某固定点$O$的动量矩定理}
\begin{itemize}
    \item 动量矩 \begin{itemize}
              \item 对惯性系某固定点$O$的动量矩:$\vec{J}_O=\sum_{i=1}^n \left(\vec{r}_i \times m_i \vec{v}_i\right)$
              \item 对非惯性系某固定点$O$的动量矩:$\vec{J}_O^{\prime} = \sum_{i=1}^{n} \left(\vec{r}_{i}^{\ \prime} \times m_{i} \vec{v}_{i}^{\,\prime}\right)$
              \item 两者间的关系(对于质心系):$\vec{J}_O=\vec{r}_c^{\ \prime} \times m\vec{v}_c+\vec{J}_c^{\prime}$
          \end{itemize}

\end{itemize}
\begin{table}[h]
    \begin{tabular}{c|c|c}
        \hline
               & 动能定理              & 机械能守恒定律 \\
        \hline
        惯性系 & $
            dW=dT=\sum_{i=1}^n \vec{F}_i^{(e)} \cdot d\vec{r}_i + \sum_{i=1}^n \vec{F}_i^{(i)} \cdot d\vec{r}_i
        $      & \begin{tabular}{c}
                     $T+V=E$ \\
                     $T+V^{(e)}+V^{(i)}=E$
                 \end{tabular}                   \\
        \hline
        质心系 & $
            dW^{\prime}=dT^{\prime}=\sum_{i=1}^n \vec{F}_i^{(e)} \cdot d\vec{r}_i^{\,\prime} + \sum_{i=1}^n \vec{F}_i^{(i)} \cdot d\vec{r}_i^{\,\prime}
        $
               & $
            T^{\prime}+V=E^{\prime}
        $                                               \\
        \hline
    \end{tabular}
\end{table}
两题问题的动力学方程:$\mu \ddot{\vec{r}}=-k^2\frac{m}{r^2}\vec{e}_r$\\
变质量物体的动力学方程:$m\frac{d\vec{v}}{dt}+\frac{dm}{dt}\vec{v}-\frac{dm}{dt}\vec{u}=\vec{F}$或$\frac{d(m\vec{v})}{dt}-\vec{u}\frac{dm}{dt}=\vec{F}$


\end{document}